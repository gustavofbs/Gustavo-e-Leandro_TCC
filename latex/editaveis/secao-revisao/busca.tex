\section{Realização da Busca e Seleção dos Artigos}

Com a string de busca definida, foram estabalecidos os critérios de inclusão e exclusão com o propósito de selecionar apenas os estudos mais aderentes ao escopo desta pesquisa, utilizando a base de dados Scopus \citeonline{elsevier2022}. Inicialmente, procedeu-se à leitura de títulos, resumos e palavras-chave de todos os trabalhos retornados pela busca, etapa fundamental para realizar uma triagem preliminar mais rigorosa e identificar estudos que, ainda que semanticamente próximos, não contribuíam efetivamente para os objetivos do trabalho. Posteriormente, os artigos que atenderam aos critérios estabelecidos foram lidos integralmente, e os dados relevantes para responder às questões de pesquisa foram organizados e registrados em um formulário de extração, assegurando consistência na análise. O protocolo consolidado pode ser visto nas tabelas \ref{tab:questoes-pesquisa}, \ref{tab:criterios} e \ref{tab:form-extracao}. 

As questões apresentadas na Tabela \ref{tab:questoes-pesquisa} não configuram objetivos independentes da pesquisa, mas representam desdobramentos operacionais da questão principal do estudo. Elas foram formuladas apenas para orientar o processo de extração e análise dos artigos selecionados.

Ao todo, a execução da string de busca realizada em 09 de outubro de 2025 resultou em 433 artigos, dos quais 22 foram selecionados para leitura completa após a aplicação criteriosa dos filtros definidos. Essa etapa permitiu refinar o conjunto inicial, privilegiando estudos que tratam diretamente da integração entre BDD, modelos de linguagem, extração semântica e automação de testes.

O conjunto final de artigos selecionados que podem ser vistos na Tabela \ref{tab:artigos-selecionados} refletiu, portanto, um equilíbrio entre o rigor metodológico da busca automatizada, permitindo contemplar diferentes perspectivas relevantes ao domínio investigado. A sequência das etapas realizadas, desde a filtragem inicial até a seleção final, foi conduzida de forma a garantir transparência, alinhamento com o protocolo de pesquisa e aderência às questões que orientam este trabalho.

\begin{table}
\centering
\caption{Questões de Pesquisa}
\label{tab:questoes-pesquisa}
\begin{tabular}{|p{4cm}|p{11cm}|}
\hline
\textbf{Questões de Pesquisa} & 
\begin{enumerate}
    \item De que maneira as especificações BDD devem ser escritas e detalhadas para permitir identificação automática de elementos de interface por IA?
    \item Quais padrões e boas práticas reduzem ambiguidades em cenários BDD?
    \item Quais desafios técnicos a IA enfrenta ao interpretar especificações BDD?
    \item Que técnicas de NLP têm sido aplicadas em interpretação automática de BDD?
    \item Como a padronização de nomenclatura e estrutura influencia a identificação de elementos?
    \item Quais ferramentas geram testes automaticamente a partir de BDD e como tratam ambiguidades?
    \item Como modelos semânticos reduzem ambiguidades na interpretação de linguagem natural?
    \item Como a complexidade da interface gráfica afeta a precisão da IA na identificação de elementos?
    \item Como estilos de escrita (natural vs técnica) influenciam a interpretação automática?
    \item Quais fatores limitam a adoção industrial de IA para transformar BDD em testes automatizados?
\end{enumerate}
\\ \hline
\end{tabular}
\begin{center}\caption*{Fonte: Autor}\end{center}
\end{table}



\begin{table}
\centering
\caption{Critérios de Inclusão e Exclusão}
\label{tab:criterios}
\begin{tabular}{|p{4cm}|p{11cm}|}
\hline
\textbf{Critérios de Inclusão} &
\begin{enumerate}
    \item Estudos publicados entre 2022 e 2025.
    \item Testes funcionais em nível de sistema que utilizem BDD.
    \item Estudos sobre automação de testes relacionados a BDD ou critérios de aceitação.
    \item Trabalhos sobre BDD ou abordagens baseadas em cenários.
    \item Publicações em inglês ou português.
\end{enumerate}
\\ \hline

\textbf{Critérios de Exclusão} &
\begin{enumerate}
    \item Estudos sem relação com BDD ou testes funcionais de sistema.
    \item Estudos focados apenas em ferramentas de automação.
    \item Trabalhos duplicados.
    \item Estudos fora da Scopus, IEEE Xplore ou ACM Digital Library.
\end{enumerate}
\\ \hline
\end{tabular}
\begin{center}\caption*{Fonte: Autor}\end{center}
\end{table}


\begin{table}
\centering
\caption{Formulário de Extração de Dados}
\label{tab:form-extracao}

\begin{tabular}{|p{4cm}|p{11cm}|}
\hline
\textbf{Campos de Extração} &
Q1. Título \newline
Q2. Resumo \newline
Q3. Ano de Publicação \newline
Q4. Fonte de Publicação \newline
Q5. Autores \newline
Q6. Palavras-chave \newline
Q7. Evidências relacionadas às QP1–QP10 \newline
\quad - QP1: Escrita/detalhamento de BDD \newline
\quad - QP2: Boas práticas de BDD \newline
\quad - QP3: Desafios técnicos da IA \newline
\quad - QP4: Técnicas de NLP \newline
\quad - QP5: Padronização de cenários \newline
\quad - QP6: Ferramentas de automação \newline
\quad - QP7: Modelos semânticos \newline
\quad - QP8: Complexidade da interface \newline
\quad - QP9: Estilos de escrita \newline
\quad - QP10: Fatores limitadores
\\ \hline
\end{tabular}
\begin{center}\caption*{Fonte: Autor}\end{center}
\end{table}

\begin{longtable}{|p{1cm}|p{4cm}|p{3cm}|p{3cm}|p{4cm}|}
    \caption{Artigos Selecionados} \label{tab:artigos-selecionados} \\
    \hline
    \textbf{Nº} & \textbf{Título} & \textbf{Publicado em Revista} & \textbf{Validação Experimental} & \textbf{Referência} \\
    \hline
    \endfirsthead
    
    \multicolumn{5}{c}{\tablename\ \thetable{} -- Continuação} \\
    \hline
    \textbf{Nº} & \textbf{Título} & \textbf{Publicado em Revista} & \textbf{Validação Experimental} & \textbf{Referência} \\
    \hline
    \endhead
        
    \hline
    \multicolumn{5}{c}{Fonte: Autor} \\
    \endlastfoot

1 & \textit{A Behavior-driven Development and Reinforcement Learning approach for videogame automated testing} & Não & Sim & \cite{Mastain2024} \\ \hline
2 & \textit{AI Driven Tools in Modern Software Quality Assurance: An Assessment of Benefits, Challenges and Future Directions} & Sim & Não & \cite{Pysmennyi2025} \\ \hline
3 & \textit{AI-Generated Test Scripts for Web E2E Testing with ChatGPT and Copilot: A Preliminary Study} & Não & Sim & \cite{Leotta2024} \\ \hline
4 & \textit{Acceptance Test Generation with Large Language Models: An Industrial Case Study} & Não & Sim & \cite{Ferreira2025} \\ \hline
5 & \textit{Agentic AI for Behavior-Driven Development Testing Using Large Language Models} & Não & Não & \cite{Paduraru2025} \\ \hline
6 & \textit{Agile testing using user language automation with artificial intelligence in Enjisset} & Não & Não & \cite{Penagos2024} \\ \hline
7 & \textit{An Autonomous RL Agent Methodology for Dynamic Web UI Testing in a BDD Framework} & Sim &Sim & \cite{Mughal2025} \\ \hline
8 & \textit{Assessing the Quality of Behavior-Driven Development Scenarios Using BERT} & Não & Não & \cite{Alinezhadtilaki2025} \\ \hline
9 & \textit{Automatic Generation of Acceptance Test Cases from Use Case Specifications: An NLP-Based Approach} & Sim & Sim & \cite{Wang2022} \\ \hline
10 & \textit{Comprehensive Evaluation and Insights Into the Use of Large Language Models in the Automation of Behavior-Driven Development Acceptance Test Formulation} & Sim & Não & \cite{Karpurapu2024} \\ \hline
11 & \textit{Cypress Copilot: Development of an AI Assistant for Boosting Productivity and Transforming Web Application Testing} & Sim & Não & \cite{Nettur2025} \\ \hline
12 & \textit{Empowering Agile-Based Generative Software Development through Human-AI Teamwork} & Sim & Sim & \cite{Zhang2025} \\ \hline
13 & \textit{Exploring LLMs Impact on Student-Created User Stories and Acceptance Testing in Software Development} & Não & Sim & \cite{Brockenbrough2025} \\ \hline
14 & \textit{Exploring Large Language Models for Requirements on String Values} & Não & Não & \cite{Babikian2025} \\ \hline
15 & \textit{Generating Multiple Conceptual Models from Behavior-Driven Development Scenarios} & Sim & Não & \cite{GUPTA2023102141} \\ \hline
16 & \textit{Object Oriented BDD and Executable Human-Language Module Specification} & Não & & \cite{Lee2023} \\ \hline
17 & \textit{Prompting Creative Requirements via Traceable and Adversarial Examples in Deep Learning} & Não & Não & \cite{Gudaparthi2023} \\ \hline
18 & \textit{RiverGame-a game testing tool using artificial intelligence} & Não & Sim & \cite{Paduraru2022} \\ \hline
19 & \textit{Test Case Prioritization Based on Neural Network Classification with Artifacts Traceability} & Não & Não & \cite{Rotaru2023} \\ \hline
20 & \textit{The Impact of Generative AI on Test \& Evaluation: Challenges and Opportunities} & Não & Não & \cite{Freeman2025} \\ \hline
21 & \textit{Towards Code Generation from BDD Test Case Specifications: A Vision} & Não & Não & \cite{Chemnitz2023} \\ \hline
22 & \textit{Visual Test Framework: Enhancing Software Test Automation with Visual Artificial Intelligence and Behavioral Driven Development} & Não & Não & \cite{Ragel2023} \\ \hline

\end{longtable}






