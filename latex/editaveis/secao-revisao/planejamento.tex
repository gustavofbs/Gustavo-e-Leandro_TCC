\section{Planejamento e Protocolo}
\label{sec:rsl-protocolo}

Inicialmente, foram realizadas leituras de estudos relacionados à automação de testes Behavior-Driven Development (BDD) com técnicas de Processamento de Linguagem Natural (PLN) e modelos de linguagem, de modo a caracterizar a literatura em relação ao tema. Esse conjunto preliminar de trabalhos teve como objetivo contextualizar o autor quanto aos principais conceitos envolvidos na integração entre IA generativa, recuperação semântica e frameworks de teste. As leituras incluíram pesquisas que abordam desde a extração semântica de passos Given/When/Then até arquiteturas que aplicam agentes autônomos e grandes modelos de linguagem (LLMs) na criação ou interpretação de cenários de teste.

Após essa etapa inicial, optou-se por selecionar determinados artigos para atuarem como estudos de controle durante o processo de construção e validação da estratégia de busca. A função desses estudos foi garantir que a string de busca elaborada fosse capaz de recuperar, de forma consistente, trabalhos representativos do domínio investigado, permitindo observar como pequenas variações nos termos utilizados afetavam os resultados obtidos. Essa escolha visou a obtenção de publicações que sintetizam, discutem ou demonstram tecnologias diretamente relacionadas à proposta deste trabalho, como, por exemplo, o uso de LLMs em pipelines de BDD e o emprego de agentes inteligentes na automação de testes \cite{Paduraru2025} e abordagens que integram exploração autônoma e BDD em ambientes web dinâmicos \cite{Mughal2025}. Esses artigos funcionaram como referências centrais para avaliar a precisão e abrangência da busca.

Definidos os estudos de controle, partiu-se então para a formalização do protocolo de pesquisa. Embora o presente trabalho não constitua uma revisão sistemática estrita, adotou-se a estrutura metodológica proposta por \cite{kitchenham2007guidelines}, de forma a assegurar replicabilidade, rastreabilidade e transparência nos procedimentos adotados. O principal objetivo da busca estruturada foi identificar produções científicas que tratassem da interseção entre BDD, modelos de linguagem, extração semântica e automação de testes por agentes, possibilitando reunir um conjunto coerente de trabalhos alinhados às questões de pesquisa deste TCC.

O protocolo considerou, portanto, critérios explícitos de inclusão e exclusão, bases específicas para consulta e uma string de busca construída a partir dos termos e conceitos observados nos estudos de controle. A partir da aplicação desse protocolo, o conjunto final de trabalhos selecionados concentrou-se em pesquisas que descrevem abordagens baseadas em PLN, embeddings, RAG e LLMs para interpretação ou geração de cenários de teste, permitindo consolidar a fundamentação teórica necessária para o desenvolvimento da solução proposta. Dessa forma, a metodologia empregada garantiu que a revisão da literatura fosse conduzida de maneira focada, criteriosa e alinhada à natureza técnico-experimental do projeto.