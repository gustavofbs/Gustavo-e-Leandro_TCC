\begin{resumo}[Abstract]
  The increasing complexity of graphical user interfaces, mainly driven by the evolution of frontend frameworks, has intensified the challenges faced by software test automation tools, which have traditionally been code-based, as well as by more recent approaches that incorporate or rely on Artificial Intelligence to support test generation. In modern applications, identifying and accessing graphical interface elements has become increasingly difficult, as identifiers are often ambiguous, dynamic, or unstable, which compromises the automatic generation of test code and results in defective or unreliable tests. In this context, different Artificial Intelligence–based strategies, such as the use of Large Language Models (LLMs), Natural Language Processing (NLP) techniques, and AI agents, have been explored to mitigate issues related to selector identification and to facilitate the generation of automated tests. Given this scenario, this study aims to investigate the activities and techniques required to develop a system capable of receiving Behavior-Driven Development (BDD) specifications and generating automated test code with the support of Artificial Intelligence. To achieve this objective, an exploratory bibliographic study was conducted, guided by relevant literature in the field. As a continuation of the research, a retrospective observational study was planned, following the case study protocol of the CAPJU project, in order to examine the application of these techniques in addressing challenges associated with automated test generation. The analysis enables the discussion of the implications of the investigated approach, its limitations, threats to validity, and opportunities for future work, contributing to the advancement of test automation practices.

\textbf{Palavras-chave:} Keywords: Test Automation; Artificial Intelligence; Behavior-Driven Development; Automatic Code Generation; Graphical User Interfaces.
\end{resumo}
