\begin{resumo}
    O aumento da complexidade das interfaces gráficas, impulsionado principalmente pela evolução dos frameworks frontend, tem intensificado os desafios enfrentados pelas ferramentas de automação de testes de software, tradicionalmente baseadas em código, bem como por abordagens mais recentes que incorporam ou utilizam Inteligência Artificial para apoiar a geração de testes. Em aplicações modernas, a identificação e o acesso aos elementos da interface gráfica tornam-se cada vez mais difíceis, uma vez que os identificadores são frequentemente ambíguos, dinâmicos ou pouco estáveis, o que compromete a geração automática de código de teste e resulta em testes defeituosos ou pouco confiáveis. Nesse contexto, diferentes estratégias baseadas em Inteligência Artificial, como o uso de modelos de linguagem de larga escala (LLMs), técnicas de Processamento de Linguagem Natural (PLN) e agentes de IA, têm sido exploradas com o objetivo de mitigar os problemas relacionados à identificação de seletores e facilitar a geração de testes automatizados. Diante desse cenário, este estudo teve como objetivo compreender as atividades e técnicas necessárias para a criação de um sistema capaz de receber especificações em BDD e gerar código de teste automatizado com apoio de Inteligência Artificial. Para alcançar esse objetivo, foi conduzido um levantamento bibliográfico de caráter exploratório, orientado pela literatura relevante da área, e, como desdobramento da pesquisa, foi planejado um estudo observacional em retrospectiva, seguindo o protocolo de estudo de caso do projeto CAPJU, com o intuito de investigar a aplicação dessas técnicas na resolução dos desafios associados à geração automática de testes. A análise realizada permite discutir as implicações da abordagem investigada, suas limitações, ameaças à validade e oportunidades para trabalhos futuros, contribuindo para o avanço das práticas de automação de testes.

\textbf{Palavras-chave:} Automação de Testes; Inteligência Artificial; Behavior-Driven Development; Geração Automática de Código; Interfaces Gráficas.
\end{resumo}
