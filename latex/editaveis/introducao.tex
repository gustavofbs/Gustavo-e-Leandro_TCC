\chapter{Introdução}   
\label{ch:intro}

Neste capítulo são apresentados os conceitos fundamentais que norteiam este trabalho: experimentação em Engenharia de Software, incluindo métodos empíricos e revisão estruturada da literatura; modelos de qualidade de software, com ênfase na ISO/IEC 25010 e na subcaracterística testabilidade; fundamentos de Verificação e Validação e os desafios associados à automação de testes em aplicações modernas; o uso de especificações Behavior-Driven Development (BDD) como artefato de suporte à testabilidade; e, por fim, a aplicação de técnicas de Inteligência Artificial para interpretação de cenários e geração de testes automatizados. Adicionalmente, são apresentados o escopo do problema, a questão de pesquisa que orienta esta investigação e a estrutura geral do documento.
\section{Contexto}\label{contextualizacao}
aabc

\section{Problema}

Consolidar a qualidade em aplicações modernas tem se tornado cada vez mais desafiadora, especialmente diante de ciclos acelerados de desenvolvimento e da crescente complexidade das interfaces gráficas \cite{Pysmennyi_2025}. Mesmo que a automação de testes seja reconhecida como essencial para manter a confiabilidade dos sistemas, a criação e manutenção de testes End-to-End (E2E) continuam sendo tarefas custosas \cite{Leotta2024}, sujeitas a fragilidade ("flaky") e alta necessidade de intervenção humana. Com a automação, os scritps tornam-se obsoletos com frequência, e a atualização contínua de locators de interface contribui para ampliar o esforço de manutenção imposto às equipes \cite{Pysmennyi_2025}.

Em relação ao desenvolvimento ágil, a utilização de especificações BDD em Linguagem Natural busca promover um entendimento compartilhado dos requisitos. Segundo \citeonline{Alinezhadtilaki2025}, a elaboração desses cenários é frequentemente subjetiva, variável e dependente do estilo de escrita dos envolvidos. A ausência de padronização dificulta a transformação desses artefatos em testes executáveis e amplia a lacuna entre requisitos descritos e sua verificação automatizada. Além disso, avanços recentes em Inteligência Artificial, especialmente no uso de modelos de linguagem para gerar testes a partir de requisitos textuais, têm apresentado resultados promissores, mas ainda heterogêneos e com limitações quanto à executabilidade, consistência semântica e capacidade de interpretar corretamente elementos da interface \cite{Pysmennyi_2025}.

Esse cenário evidencia a lacuna de que não existem diretrizes consolidadas que orientem como especificações BDD devem ser escritas e detalhadas para favorecer a interpretação automática por sistemas baseados em IA \cite{GUPTA2023102141}. A falta de um modelo claro que minimize ambiguidades e promova uma correspondência precisa entre os passos descritos e os elementos reais da interface compromete a confiabilidade dos testes gerados e dificulta a adoção confiável dessas tecnologias \cite{Babikian2025}. Assim, torna-se necessário investigar em que medida a forma de escrita de cenários BDD influencia a capacidade de sistemas de IA em identificar corretamente os elementos da interface gráfica e produzir testes automatizados de forma precisa.

\section{Questão de Pesquisa} 

\section{Objetivos}
\label{subsec:objetivos-pesquisa}

O principal objetivo desta monografia é identificar como a escrita nas especificações BDD influenciam a capacidade de modelos baseados em Inteligência Artificial de identificar, de forma autônoma, elementos de interface gráfica necessários para gerar testes de sistema automatizados. Para alcançar esse propósito, foram definidos os seguintes objetivos específicos:

\begin{itemize} 
    \item Fundamentar os conceitos relacionados a Engenharia de Software Experimental, modelos de qualidade de software, princípios da Metodologia Ágil, Behavior-Driven Development (BDD) e Processamento de Linguagem Natural (NLP), destacando fatores que influenciam na interpretação por parte de sistemas de IA; 
    \item Analisar padrões, boas práticas e desafios associados à escrita de cenários BDD, com foco na redução de ambiguidades que dificultam a interpretação automática por sistemas de IA; 
    \item Identificar limitações e capacidades de modelos de IA e técnicas de NLP na interpretação de requisitos textuais e na identificação de elementos de interface gráfica para suporte à geração de testes automatizados;
    \item Planejar um estudo observacional em retrospectiva, que permita investigar o uso de modelos de linguagem de larga escala para apoiar a escrita de casos de teste em cenários BDD. Esse estudo será executado ao longo do período do trabalho de conclusão de curso 2.
\end{itemize}

\input{editaveis/secao-introducao/metodologia}

\section{Organização do Trabalho}
\label{sec:organizacao}

A seguir, são apresentados os capítulos que compõem a estrutura desta monografia.

\begin{itemize}
    \item \textbf{Introdução:} apresentação do contexto da pesquisa, o problema investigado, a formulação da questão de pesquisa, os objetivos geral e específicos e a justificativa do estudo. Por fim, descreve a organização geral do trabalho e as atividades realizadas na primeira etapa.;
    \item \textbf{Referencial Teórico:} fundamentação teórica para o desenvolvimento da pesquisa, abordando Engenharia de Software Experimental, metodologias ágeis, modelos de qualidade, além dos princípios do Behavior-Driven Development (BDD) e de técnicas de Inteligência Artificial e Processamento de Linguagem Natural (NLP).;
    \item \textbf{Revisão Estruturada da Literatura:} metodologia empregada para identificação, seleção e análise dos estudos que embasam esta monografia, incluindo a definição do protocolo de pesquisa e o mapeamento dos resultados encontrados.;
    \item \textbf{Proposta do Estudo de Caso:} descrição do protocolo utilizado para a condução do estudo de caso, apresentando seus objetivos, perguntas de pesquisa, etapas executadas e os principais resultados obtidos com a aplicação da abordagem proposta neste trabalho.;
    \item \textbf{Conclusão:} apresentação das considerações finais do trabalho, discutindo os principais resultados alcançados, as contribuições geradas, as limitações encontradas e sugestões para pesquisas futuras relacionadas ao tema.
\end{itemize}

\section{Cronograma e Fluxo de Atividades}\label{cronograma}

\subsection{Atividades da Primeira Etapa}
\label{cronograma1}

Esta subseção tem como função descrever as atividades realizadas na primeira etapa do trabalho, representando-o em forma de fluxo na Figura \ref{fig:atividades_1} e em cronograma na Figura \ref{fig:cronograma_1}.

\begin{figure}[h!]
    \caption{Fluxo de Atividades na Primeira Etapa da Monografia}
    \centering
    \includegraphics[width=.7\linewidth]{figuras/atividades1.png}
    \caption*{Fonte: Autor}
    \label{fig:atividades_1}
\end{figure}
             
\begin{figure}[h!]
    \caption{Cronograma das Atividades na Primeira Etapa da Monografia}
    \centering
    \includegraphics[width=.7\linewidth]{figuras/cronograma1.png}
    \caption*{Fonte: Autor}
    \label{fig:cronograma_1}
\end{figure}


\begin{itemize}
    \item \textbf{Contextualização em Experimentação na Engenharia de Software:} estudo relacionado aos métodos de pesquisa empírica aplicados à Engenharia de Software, incluindo revisão estruturada da literatura, experimentos, survey e estudo de caso. Esses conceitos fornecem a base metodológica necessária para apoiar a condução da investigação proposta neste trabalho \cite{Wohlin:2012:ESE:2349018};
    \item \textbf{Definição do GQM:} definição do protocolo metodológico adotado na pesquisa através da abordagem Goal Question Metric (GQM) \cite{gqm} para a formulação da questão de pesquisa, de suas subquestões e das métricas a serem analisadas;
    \item \textbf{Definição do Protocolo de Revisão da Literatura:} formulação e adaptação do protocolo de revisão da literatura com base nas diretrizes de \citeonline{kitchenham2007guidelines}, contemplando a definição das questões de pesquisa da revisão, a construção da string de busca estruturada com o apoio do protocolo PICO \cite{Pai2004}, além do estabelecimento dos critérios de inclusão e exclusão e do método utilizado para análise e extração dos dados dos estudos selecionados;
    \item \textbf{Execução da String de Busca e Seleção dos Artigos:} seleção dos estudos obtidos a partir da execução da string de busca, realizando a triagem inicial por meio da leitura de títulos, resumos e palavras-chave, conforme os critérios de inclusão e exclusão definidos. Quando necessário, procede-se à leitura parcial dos artigos para confirmar sua relevância e compor o conjunto final de trabalhos analisados;
    \item \textbf{Leitura do Material Selecionado:} leitura completa dos estudos selecionados, com o objetivo de aprofundar o entendimento sobre o estado da arte, identificar abordagens, modelos e práticas já existentes e consolidar o conhecimento necessário para embasar a formulação da proposta deste trabalho;
    \item \textbf{Definição da Proposta de Solução:} apresentação da abordagem desenvolvida neste trabalho, detalhando o processo adotado, os artefatos envolvidos e as atividades previstas para o estudo de caso;
    \item \textbf{Escrita da Monografia:} redação da monografia seguindo a organização apresentada na Seção \ref{sec:organizacao};
    \item \textbf{Revisão da Monografia:} realização das correções e ajustes solicitados pelo orientador;
    \item \textbf{Apresentação da Monografia:} preparação e apresentação da monografia proposta para a banca avaliadora.
\end{itemize}
