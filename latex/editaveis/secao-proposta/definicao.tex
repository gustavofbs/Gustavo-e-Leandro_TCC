\section{Definição}

aa

\section{Objetivo}

Segundo pesquisas recentes sobre o uso de modelos de linguagem e técnicas de Inteligência Artificial no apoio ao desenvolvimento e à automação de testes \cite{Ferreira2025, Nettur2025}, práticas baseadas em Behavior-Driven Development (BDD) têm se consolidado como um meio eficaz para expressar o comportamento esperado de sistemas por meio de linguagem natural estruturada. Contudo, a forma como esses cenários são escritos em especial a clareza, o nível de detalhamento e a padronização terminológica exerce influência direta sobre a capacidade de sistemas baseados em IA interpretarem corretamente as ações descritas e identificarem, de maneira autônoma, os elementos de interface gráfica necessários para a automação de testes.

Entretanto, observa-se na prática que especificações BDD frequentemente apresentam inconsistências, variações estruturais e ambiguidades que dificultam sua interpretação automatizada, limitando o potencial das ferramentas de IA para apoiar atividades como geração de testes, identificação de seletores e mapeamento de elementos da interface. Essa lacuna se torna ainda mais relevante diante do avanço de soluções que dependem da compreensão semântica de cenários para gerar testes de sistema de forma autônoma.

Dessa forma, o objetivo deste estudo consiste em investigar de que maneira as especificações BDD devem ser escritas, estruturadas e detalhadas para que sistemas baseados em Inteligência Artificial consigam identificar, de forma autônoma, os elementos de interface gráfica necessários à criação de testes de sistema automatizados. Para alcançar esse objetivo, serão consideradas abordagens, técnicas e ferramentas discutidas na literatura especializada, bem como práticas adotadas na indústria, incluindo o uso de processos automatizados de análise e interpretação textual capazes de estruturar, enriquecer e comparar semanticamente cenários BDD, com o intuito de compreender quais características textuais e estruturais favorecem a correta interpretação dos cenários por sistemas de IA.