\section{Problema}

Consolidar a qualidade em aplicações modernas tem se tornado cada vez mais desafiadora, especialmente diante de ciclos acelerados de desenvolvimento e da crescente complexidade das interfaces gráficas \cite{Pysmennyi_2025}. Mesmo que a automação de testes seja reconhecida como essencial para manter a confiabilidade dos sistemas, a criação e manutenção de testes End-to-End (E2E) continuam sendo tarefas custosas \cite{Leotta2024}, sujeitas a fragilidade ("flaky") e alta necessidade de intervenção humana. Com a automação, os scritps tornam-se obsoletos com frequência, e a atualização contínua de locators de interface contribui para ampliar o esforço de manutenção imposto às equipes \cite{Pysmennyi_2025}.

Em relação ao desenvolvimento ágil, a utilização de especificações BDD em Linguagem Natural busca promover um entendimento compartilhado dos requisitos. Segundo \citeonline{Alinezhadtilaki2025}, a elaboração desses cenários é frequentemente subjetiva, variável e dependente do estilo de escrita dos envolvidos. A ausência de padronização dificulta a transformação desses artefatos em testes executáveis e amplia a lacuna entre requisitos descritos e sua verificação automatizada. Além disso, avanços recentes em Inteligência Artificial, especialmente no uso de modelos de linguagem para gerar testes a partir de requisitos textuais, têm apresentado resultados promissores, mas ainda heterogêneos e com limitações quanto à executabilidade, consistência semântica e capacidade de interpretar corretamente elementos da interface \cite{Pysmennyi_2025}.

Esse cenário evidencia a lacuna de que não existem diretrizes consolidadas que orientem como especificações BDD devem ser escritas e detalhadas para favorecer a interpretação automática por sistemas baseados em IA \cite{GUPTA2023102141}. A falta de um modelo claro que minimize ambiguidades e promova uma correspondência precisa entre os passos descritos e os elementos reais da interface compromete a confiabilidade dos testes gerados e dificulta a adoção confiável dessas tecnologias \cite{Babikian2025}. Assim, torna-se necessário investigar em que medida a forma de escrita de cenários BDD influencia a capacidade de sistemas de IA em identificar corretamente os elementos da interface gráfica e produzir testes automatizados de forma precisa.