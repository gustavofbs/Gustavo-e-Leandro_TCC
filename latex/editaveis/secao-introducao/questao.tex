\section{Questão de Pesquisa} 
\label{sec:questao}

A definição da questão de pesquisa central da monografia foi realizada com o apoio da abordagem \textit{Goal Question Metric} (GQM) , que estrutura de maneira \textit{top-down} a relação entre os objetivos de uma investigação, as perguntas formuladas para alcançar tais objetivos e as métricas necessárias afim de obter respostas quantitativas para respondê-las, podendo ser vista na Figura \ref{fig:gqm_adaptado} \cite{Basili1994}. A estrutura do GQM foi adaptada ao contexto deste estudo, contemplando a definição do propósito, objeto de análise, ponto de vista e foco da avaliação, conforme a Tabela \ref{tab:gqm}.

\begin{table}[h]
    \centering
	\caption{GQM Adaptado}
    \begin{tabular}{|p{5cm}|p{9cm}|}
        \hline
        \textbf{Característica} & \textbf{Valor} \\
        \hline
        \textbf{Analisar} & a característica de qualidade de produto de software de  Manutenibilidade, em particular,  sua subcaracterística de Testabilidade, sob a perspectiva da qualidade interna. \\ 
        \hline
        \textbf{Com o propósito de} & caracterizar \\
        \hline
        \textbf{Com respeito ao} & utilização de especificações BDD em conjunto com técnicas de Inteligência Artificial para identificação de elementos da interface gráfica e geração de testes automatizados. \\
        \hline
        \textbf{Do ponto de vista de} & Pesquisador \\
        \hline
        \textbf{No contexto de} & projetos de software livre, com código aberto em plataformas colaborativas como GitHub\\
        \hline
    \end{tabular} \\[0.5em]
   	Fonte: Adaptado de \citeonline{Basili1994}
	    \label{tab:gqm}
\end{table}

Ao seguirmos essa estrutura, definimos a questão de pesquisa central deste trabalho, que é apresentada a seguir:

\begin{center}
    \textit{De que maneira as especificações BDD devem ser escritas e detalhadas para que sistemas baseados em IA consigam identificar, de forma autônoma, os elementos de interface gráfica necessários para criar testes de sistema automatizados?}   
\end{center}

\begin{figure}[h!]
    \caption{Estrutura do Modelo GQM}
    \centering
    \includegraphics[width=.7\linewidth]{figuras/gqm_adaptado.png}
    \caption*{Fonte: Adaptado de \citeonline{Basili1994}}
    \label{fig:gqm_adaptado}
\end{figure}