\section{Objetivos}
\label{subsec:objetivos-pesquisa}

O objetivo desta monografia consiste em investigar como as especificações BDD devem ser escritas e estruturadas para permitir que sistemas baseados em Inteligência Artificial identifiquem, de forma autônoma, elementos de interface gráfica necessários para gerar testes de sistema automatizados.

Para alcançar esse propósito, foram definidos os seguintes objetivos específicos::

\begin{itemize} 
    \item Fundamentar os conceitos relacionados a Engenharia de Software Experimental, modelos de qualidade de software, princípios da Metodologia Ágil, Behavior-Driven Development (BDD) e Processamento de Linguagem Natural (NLP), destacando fatores que influenciam nas especificações em linguagem natural; 
    \item Analisar padrões, boas práticas e desafios associados à escrita de cenários BDD, com foco na redução de ambiguidades que dificultam a interpretação automática por sistemas de IA; 
    \item Caracterizar limitações e capacidades de modelos de IA e técnicas de NLP na interpretação de requisitos textuais e na identificação de elementos de interface gráfica para suporte à geração de testes automatizados;
    \item Propor e estruturar uma abordagem experimental, envolvendo o uso de uma LLM para produzir testes de sistema automatizados a partir de especificações BDD, permitindo observar como variações na escrita impactam a interpretação da IA;
    \item Planejar um estudo de caso que permita comparar a abordagem baseada em IA com uma ferramenta tradicional de automação de testes, avaliando os efeitos das diferenças entre estilos e padrões de escrita em BDD 
    \item Conduzir o estudo de caso, realizando a coleta das métricas selecionadas como precisão da identificação de elementos, executabilidade, relevância semântica e cobertura dos critérios de aceitação.
    \item Interpretar e discutir os resultados, identificando quais características das especificações influenciam diretamente o desempenho da IA na geração de testes automatizados.
    \item Registrar as conclusões deste trabalho, sintetizando os achados da pesquisa, suas limitações e as oportunidades para investigações futuras.
\end{itemize}