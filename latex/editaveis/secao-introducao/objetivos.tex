\section{Objetivos}
\label{subsec:objetivos-pesquisa}

O principal objetivo desta monografia é identificar como a escrita nas especificações BDD influenciam a capacidade de modelos baseados em Inteligência Artificial de identificar, de forma autônoma, elementos de interface gráfica necessários para gerar testes de sistema automatizados. Para alcançar esse propósito, foram definidos os seguintes objetivos específicos:

\begin{itemize} 
    \item Fundamentar os conceitos relacionados a Engenharia de Software Experimental, modelos de qualidade de software, princípios da Metodologia Ágil, Behavior-Driven Development (BDD) e Processamento de Linguagem Natural (NLP), destacando fatores que influenciam na interpretação por parte de sistemas de IA; 
    \item Analisar padrões, boas práticas e desafios associados à escrita de cenários BDD, com foco na redução de ambiguidades que dificultam a interpretação automática por sistemas de IA; 
    \item Identificar limitações e capacidades de modelos de IA e técnicas de NLP na interpretação de requisitos textuais e na identificação de elementos de interface gráfica para suporte à geração de testes automatizados;
    \item Planejar um estudo observacional em retrospectiva, que permita investigar o uso de modelos de linguagem de larga escala para apoiar a escrita de casos de teste em cenários BDD. Esse estudo será executado ao longo do período do trabalho de conclusão de curso 2.
\end{itemize}