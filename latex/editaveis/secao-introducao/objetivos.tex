\section{Objetivos}
\label{subsec:objetivos-pesquisa}

O objetivo desta monografia consiste em investigar como as especificações BDD devem ser escritas e estruturadas para permitir que sistemas baseados em Inteligência Artificial identifiquem, de forma autônoma, elementos de interface gráfica necessários para gerar testes de sistema automatizados. 

\todo[inline, color=pink]{Reescrever o parágrafo anterior. Praticamente repetiu a questão de pesquisa}


Para alcançar esse propósito, foram definidos os seguintes objetivos específicos::

\begin{itemize} 
    \item Fundamentar os conceitos relacionados a Engenharia de Software Experimental, modelos de qualidade de software, princípios da Metodologia Ágil, Behavior-Driven Development (BDD) e Processamento de Linguagem Natural (NLP), destacando fatores que influenciam nas especificações em \textcolor{blue}{linguagem natural}\todo[color=yellow]{Alinhar com as correções relacionadas anteriormente}l; 
    \item Analisar padrões, boas práticas e desafios associados à escrita de cenários BDD, com foco na redução de ambiguidades que dificultam a interpretação automática por sistemas de IA; 
    \item Identificar limitações e capacidades de modelos de IA e técnicas de NLP na interpretação de requisitos textuais e na identificação de elementos de interface gráfica para suporte à geração de testes automatizados;
     \item Planejar uma  um estudo observacional em retrospectiva, que permita investigar o uso de modelos de linguagem de larga escala para apoiar a escrita de casos de teste em cenários BDD. Esse estudo será executado ao longo do período do trabalho de conclusão de curso 2.
\end{itemize}