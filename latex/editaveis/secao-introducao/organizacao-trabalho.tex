\section{Organização do Trabalho}
\label{sec:organizacao}

A seguir, são apresentados os capítulos que compõem a estrutura desta monografia.

\begin{itemize}
    \item \textbf{Introdução:} apresentação do contexto da pesquisa, o problema investigado, a formulação da questão de pesquisa, os objetivos geral e específicos e a justificativa do estudo. Por fim, descreve a organização geral do trabalho e as atividades realizadas na primeira etapa;
    \item \textbf{Referencial Teórico:} fundamentação teórica para o desenvolvimento da pesquisa, abordando Engenharia de Software Experimental, metodologias ágeis, modelos de qualidade, além dos princípios do Behavior-Driven Development (BDD) e de técnicas de Inteligência Artificial e Processamento de Linguagem Natural (NLP);
    \item \textbf{Revisão Estruturada da Literatura:} metodologia empregada para identificação, seleção e análise dos estudos que embasam esta monografia, incluindo a definição do protocolo de pesquisa e o mapeamento dos resultados encontrados;
    \item \textbf{Planejamento do estudo observacional:} descrição do protocolo utilizado para a condução do estudo observacional em retrospectiva, apresentando seus objetivos, quetões de pesquisa, casos e unidades de análise a serem observados;
    \item \textbf{Conclusão:} apresentação das considerações finais do trabalho, discutindo os principais resultados alcançados, as contribuições geradas, as limitações encontradas e o prosseguimento do estudo no âmbito do trabalho de conslusão de curso II.
\end{itemize}