\section{Contexto}\label{contextualizacao}

A qualidade de software mede o quanto um sistema atende às necessidades dos usuários, considerando padrões de desenvolvimento e requisitos definidos. Esses modelos servem de base para medir, especificar e avaliar a qualidade de sistemas e softwares, tornando a verificação dos requisitos mais padronizada e consistente. A norma afirma que suas características e subcaracterísticas fornecem uma terminologia consistente que orienta a especificação, a mensuração e a avaliação da qualidade de produtos e sistemas de software \cite{ISO25010}. No modelo Quality in Use, a qualidade é avaliada conforme a interação do usuário com o sistema. Ele possui cinco características principais: efetividade, eficiência, satisfação, ausência de riscos e cobertura de contexto \cite{ISO25010}. Já o modelo Product Quality foca nas propriedades internas e externas do sistema. Ele contempla oito características: aptidão funcional, eficiência de desempenho, compatibilidade, usabilidade, confiabilidade, segurança, manutenibilidade e portabilidade \cite{ISO25010}. É importante que os sistemas de software sejam avaliados quanto à sua qualidade para garantir que o produto atenda aos requisitos e gere valor aos stakeholders. As características de qualidade devem ser avaliadas, medidas e especificadas sempre que possível, utilizando métodos aceitos academicamente \cite{ISO25010}. A avaliação da qualidade deve considerar propriedades internas, como arquitetura, estrutura e código, e propriedades externas, como o comportamento do sistema em execução e a qualidade em uso, que representa o impacto real no usuário. Com essas medições, é possível prever problemas, garantir que o software cumpra os requisitos definidos e tornar o processo de desenvolvimento mais confiável e estruturado.

O Modelo de McCall (1977) foi um dos primeiros modelos de qualidade de software e buscava aproximar desenvolvedores e usuários ao organizar a qualidade em três grandes perspectivas. Esse modelo foi concebido com o objetivo de reduzir a distância entre as necessidades do usuário e as prioridades técnicas do desenvolvedor \cite{AlQutaish2010}. A revisão do produto aborda a capacidade de modificação do software ao longo de sua vida útil, incluindo aspectos como facilidade de manutenção, adaptação a mudanças e possibilidade de testar o sistema de forma eficaz. A operação do produto foca no desempenho durante o uso, considerando precisão das funcionalidades, funcionamento confiável, bom uso dos recursos, segurança contra acessos indevidos e facilidade de aprendizado pelo usuário. Já a transição do produto trata da capacidade de migração ou reaproveitamento do software em novos ambientes, reunindo fatores como transferência para outras plataformas, reutilização de componentes e integração com outros sistemas, organizados em três grandes perspectivas que reúnem fatores e critérios de avaliação específicos \cite{AlQutaish2010}. O Modelo de Boehm (1978) aprofunda essa análise ao organizar a qualidade em três níveis hierárquicos. O nível de alto nível define objetivos gerais do sistema, como utilidade, manutenibilidade e portabilidade. O nível intermediário traduz esses objetivos em atributos essenciais esperados do software, atuando como elo entre as metas amplas e os aspectos técnicos. Por fim, o nível das características primitivas reúne os elementos básicos utilizados na construção das métricas que permitem avaliar a qualidade de forma quantitativa, já que o modelo foi estruturado para possibilitar uma avaliação automática e quantitativa da qualidade de software \cite{AlQutaish2010}.

Atualmente, a ISO/IEC 25010 é amplamente utilizada para avaliação da qualidade de software, e suas características e subcaracterísticas fornecem ferramentas para especificar a qualidade, possibilitando maior consistência entre requisitos de desenvolvimento. Apesar de sua relevância, ela possui uma limitação: define o que é qualidade, mas não diz como produzir ou utilizar ferramentas que apoiem essa qualidade. “Outro ponto fraco desses modelos de qualidade é que eles não especificam como os atributos de qualidade devem ser medidos e como os resultados de medição podem ser agregados para uma avaliação geral” (\citeauthor{Wagner2016}, \citeyear{Wagner2016}, tradução nossa). “Modelos como ISO 25010 descrevem e estruturam conceitos gerais que constituem software de alta qualidade. A maioria deles, porém, carece da habilidade de serem usados para avaliação ou melhoria concreta” (\citeauthor{Wagner2016}, \citeyear{Wagner2016}, tradução nossa). A norma não apresenta, por exemplo, o Behavior-Driven Development (BDD) ou qualquer metodologia de desenvolvimento que utilize testes para orientar o comportamento de um sistema, nem critérios para avaliar de forma objetiva a clareza e o propósito de cenários comportamentais. Ela também não oferece diretrizes sobre como estruturar requisitos textuais que facilitem a interpretação automática por LLMs. A própria ISO reconhece esse caráter conceitual ao afirmar que as características de qualidade devem ser especificadas, medidas e avaliadas utilizando métodos validados ou amplamente aceitos, e que o modelo serve para identificar características relevantes que posteriormente serão usadas para estabelecer requisitos, seus critérios de satisfação e medidas correspondentes \cite{ISO25010}. Assim, mesmo que a ISO seja essencial para oferecer um norte sobre qualidade, ela não fornece as ferramentas necessárias para garantir que essas expectativas possam ser operacionalizadas em ambientes modernos de automação.

Apesar da existência de padrões consolidados, a indústria ainda enfrenta diversos problemas na medição e garantia de qualidade, especialmente em sistemas de software que utilizam interfaces gráficas e dinâmicas. Muitas aplicações, principalmente web, utilizam múltiplos frameworks de frontend com elementos reativos, o que dificulta a criação de testes automatizados por meio de LLMs. “As aplicações web modernas são caracterizadas por estruturas de navegação dinâmicas e complexas, nas quais cada página e cada interação do usuário fazem parte de uma rede intricada. As metodologias tradicionais de teste muitas vezes não conseguem explorar completamente esses caminhos complexos, especialmente quando as ações do usuário são interdependentes e sensíveis ao contexto” (\citeauthor{Mughal2025}, \citeyear{Mughal2025}, tradução nossa).. A automação de testes com IA ainda enfrenta barreiras de cobertura, consistência e adaptação, pois ferramentas dependem de seletores frágeis, e os testes automatizados podem quebrar facilmente após mudanças visuais ou estruturais. Frameworks modernos de frontend podem gerar estruturas de DOM que tornam seletores e caminhos XPath instáveis, fazendo com que qualquer alteração de UI exija retrabalho para reescrever o código \cite{Pysmennyi_2025}. Os cenários de teste escritos em BDD frequentemente possuem alto nível de abstração, sem detalhes suficientes para identificar elementos da interface, o que torna a geração automática de testes uma tarefa complexa. “Cenários mal construídos podem levar a mal-entendidos, testes não confiáveis e ineficiências no processo de desenvolvimento” (\citeauthor{Alinezhadtilaki2025}, \citeyear{Alinezhadtilaki2025}, tradução nossa). Assim, medir qualidade depende diretamente da forma como os elementos e interações são descritos.

É evidente a existência de pouca literatura sobre diretrizes que orientem a elaboração de cenários em BDD para que uma IA LLM consiga identificar elementos da tela e gerar testes automatizados de sistema confiáveis. O presente trabalho busca investigar quais características textuais, estruturais e semânticas podem ser usadas ou ajustadas na escrita dos cenários BDD para favorecer o reconhecimento de elementos visuais por uma IA, possibilitando a geração de testes automatizados a partir de cenários escritos em BDD. Agentes de IA dependem de descrições mais estruturadas e de artefatos textuais explícitos para interpretar corretamente passos, elementos e estados intermediários das interfaces, enquanto métodos tradicionais como TDD e BDD não foram concebidos para lidar com sistemas baseados em LLM, pois assumem especificações estáveis e comportamentos determinísticos, o que contrasta com a natureza não determinística e adaptativa desses agentes \cite{Xia2025}.

