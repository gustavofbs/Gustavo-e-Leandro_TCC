\section{Contexto}\label{contextualizacao}

A ISO/IEC 25010 define modelos de qualidade que servem como referência para avaliar se um sistema atende às necessidades dos usuários por meio de características e subcaracterísticas bem estruturadas. Esses modelos permitem uma verificação mais consistente e padronizada da qualidade do software, auxiliando na especificação de requisitos, na avaliação das propriedades internas e externas do produto e na identificação de aspectos que impactam a experiência do usuário e o uso do sistema em diferentes contextos. Dessa forma, o processo de desenvolvimento torna-se mais controlado e alinhado aos critérios de qualidade estabelecidos pela norma \cite{ISO25010}. 

No entanto, apesar de fornecer uma base sólida para a avaliação da qualidade, a aplicação prática desses modelos ainda apresenta desafios, especialmente em sistemas modernos que utilizam interfaces gráficas dinâmicas e altamente interativas. Muitas aplicações, principalmente web, fazem uso de múltiplos frameworks de frontend com elementos reativos, o que dificulta a criação de testes automatizados por meio de LLMs. Nesse contexto, “as aplicações web modernas são caracterizadas por estruturas de navegação dinâmicas e complexas, nas quais cada página e cada interação do usuário fazem parte de uma rede intricada. As metodologias tradicionais de teste muitas vezes não conseguem explorar completamente esses caminhos complexos, especialmente quando as ações do usuário são interdependentes e sensíveis ao contexto” (\citeauthor{Mughal2025}, \citeyear{Mughal2025}, p. 1, tradução nossa).

Diante desse cenário, os processos de Verificação e Validação tornam-se ainda mais fundamentais para apoiar a garantia da qualidade de software, ao confirmar se o sistema atende aos requisitos especificados e ao uso pretendido \cite{IEEE1012-2016}. Nesse contexto, os testes de software desempenham papel central no ciclo de desenvolvimento, embora métodos tradicionais de QA enfrentem dificuldades para acompanhar a crescente complexidade e velocidade dos sistemas modernos (\citeauthor{Mastain2024}, \citeauthor{Pysmennyi2025}). Em paralelo, avanços recentes em Inteligência Artificial, especialmente com Large Language Models (LLMs) e abordagens agenticas, têm ampliado as possibilidades de automação de tarefas de teste, como a geração de casos a partir de requisitos em linguagem natural \cite{Pysmennyi2025}. Assim, a aplicação de IA generativa surge como uma alternativa promissora para complementar e modernizar práticas tradicionais, buscando melhorar cobertura, precisão e eficiência na avaliação da qualidade de software.

Apesar desse potencial, a automação de testes com IA ainda enfrenta barreiras de cobertura, consistência e adaptação, pois as ferramentas dependem de seletores identificadores usados para localizar elementos da interface, como CSS selectors ou caminhos XPath, que frequentemente são frágeis. Consequentemente, testes automatizados podem quebrar facilmente após mudanças visuais ou estruturais na aplicação. Esse problema se intensifica em frameworks modernos de frontend, como Angular e React, que geram estruturas de DOM dinâmicas; esse dinamismo torna seletores e caminhos XPath instáveis, fazendo com que qualquer alteração de UI exija retrabalho para reescrever o código (\citeauthor{Chemnitz2023}, \citeauthor{Pysmennyi2025}).

Nesse sentido, ao considerar o objetivo desta pesquisa — a geração automática de testes a partir de especificações textuais — torna-se essencial analisar como os cenários de teste são estruturados. Os cenários escritos em Behavior Driven Development (BDD) frequentemente possuem alto nível de abstração, sem detalhes suficientes para identificar elementos da interface, o que torna a geração automática de testes uma tarefa complexa. Como apontam Alinezhadtilaki e Evans (2025), “cenários mal construídos podem levar a mal-entendidos, testes não confiáveis e ineficiências no processo de desenvolvimento” (p. 1, tradução nossa).

Dessa forma, a qualidade dos cenários BDD em termos de clareza, completude, precisão e testabilidade depende diretamente da forma como os elementos e interações são descritos, uma vez que esses fatores determinam a capacidade de ferramentas automatizadas identificarem corretamente os componentes da interface e gerarem testes executáveis (\citeauthor{Alinezhadtilaki2025}, \citeauthor{Ferreira2025}).

Além disso, embora existam pesquisas relacionadas à transformação de linguagem natural em testes, são poucos os estudos focados especificamente em gerar testes automaticamente a partir de cenários BDD. A maior parte das investigações concentra-se em produzir testes para código, como classes Java ou funções Python, e não em testes sobre interfaces gráficas, os quais exigem a identificação de elementos da UI \cite{Ferreira2025}. Nesse sentido, o presente trabalho busca investigar quais características textuais, estruturais e semânticas podem ser usadas ou ajustadas na escrita dos cenários BDD para favorecer o reconhecimento de elementos visuais por uma IA, possibilitando a geração de testes automatizados a partir de cenários escritos em BDD. Agentes de IA dependem de descrições mais estruturadas e de artefatos textuais explícitos para interpretar corretamente passos, elementos e estados intermediários das interfaces, enquanto métodos tradicionais como TDD e BDD não foram concebidos para lidar com sistemas baseados em LLM, pois assumem especificações estáveis e comportamentos determinísticos, o que contrasta com a natureza não determinística e adaptativa desses agentes \cite{Xia2025}.

À luz desse contexto, o presente trabalho investiga quais características textuais, estruturais e semânticas dos cenários BDD podem ser ajustadas para tornar mais eficaz o reconhecimento de elementos visuais por sistemas de IA, de modo a viabilizar a geração automática de testes a partir desses cenários.

Essa necessidade de detalhamento adicional decorre do fato de que agentes de IA dependem de descrições mais estruturadas e de artefatos textuais explícitos para interpretar corretamente passos, elementos e estados intermediários das interfaces (\citeauthor{Mughal2025}, \citeauthor{Ferreira2025}). Tal demanda por maior formalização contrasta com a suposição de estabilidade presente em abordagens tradicionais: métodos como TDD e BDD não foram concebidos para lidar com sistemas baseados em LLM, pois assumem especificações estáveis e comportamentos determinísticos, o que difere da natureza não determinística e adaptativa desses agentes (\citeauthor{Pysmennyi2025}, \citeauthor{Freeman2025}).