\section{Cronograma e Fluxo de Atividades}\label{cronograma}

\subsection{Atividades da Primeira Etapa}
\label{cronograma1}

Nesta subseção são apresentadas e descritas as atividades realizadas na primeira etapa do trabalho, representando-o em forma de fluxo na Figura \ref{fig:atividades_1} e em cronograma na Figura \ref{fig:cronograma_1}.

\begin{figure}[h!]
    \caption{Fluxo de Atividades na Primeira Etapa da Monografia}
    \centering
    \includegraphics[width=.7\linewidth]{figuras/atividades1.png}
    \caption*{Fonte: Autor}
    \label{fig:atividades_1}
\end{figure}
             
\begin{figure}[h!]
    \caption{Cronograma das Atividades na Primeira Etapa da Monografia}
    \centering
    \includegraphics[width=.7\linewidth]{figuras/cronograma1.png}
    \caption*{Fonte: Autor}
    \label{fig:cronograma_1}
\end{figure}


\begin{itemize}
    \item \textbf{Contextualização em Experimentação na Engenharia de Software:} estudo relacionado aos métodos de pesquisa empírica aplicados à Engenharia de Software, incluindo revisão estruturada da literatura, experimentos, survey e estudo de caso. Esses conceitos fornecem a base metodológica necessária para apoiar a condução da investigação proposta neste trabalho \cite{Wohlin:2012:ESE:2349018};    
    \item \textbf{Definição do GQM:} definição do protocolo metodológico adotado na pesquisa através da abordagem Goal Question Metric (GQM) \cite{gqm} para a formulação da questão de pesquisa, de suas subquestões e das métricas a serem analisadas;
    \item \textbf{Definição do Protocolo de Revisão da Literatura:} formulação e adaptação do protocolo de revisão da literatura com base nas diretrizes de \citeonline{kitchenham2007guidelines}, contemplando a definição das questões de pesquisa da revisão, a construção da string de busca estruturada com o apoio do protocolo PICO \cite{Pai2004}, além do estabelecimento dos critérios de inclusão e exclusão e do método utilizado para análise e extração dos dados dos estudos selecionados;
    \item \textbf{Execução da String de Busca e Seleção dos Artigos:} seleção dos estudos obtidos a partir da execução da string de busca, realizando a triagem inicial por meio da leitura de títulos, resumos e palavras-chave, conforme os critérios de inclusão e exclusão definidos. Quando necessário, procede-se à leitura parcial dos artigos para confirmar sua relevância e compor o conjunto final de trabalhos analisados;
    \item \textbf{Leitura do Material Selecionado:} leitura completa dos estudos selecionados, com o objetivo de aprofundar o entendimento sobre o estado da arte, identificar abordagens, modelos e práticas já existentes e consolidar o conhecimento necessário para embasar a formulação da proposta deste trabalho;
    \item \textbf{Planejamento do Estudo Observacional:} apresentação da protocolo do estudo observacional em restrospectiva, detalhando o processo adotado, os artefatos envolvidos e as atividades previstas para a execuçãO e análise dos resultados;
    \item \textbf{Escrita da Monografia:} redação da monografia seguindo a organização apresentada na Seção \ref{sec:organizacao};
    \item \textbf{Revisão da Monografia:} realização das correções e ajustes solicitados pelo orientador;
    \item \textbf{Apresentação da Monografia:} preparação e apresentação da monografia proposta para a banca avaliadora.
\end{itemize}

\subsection{Atividades da Segunda Etapa}

Nesta subseção são apresentadas e descritas as atividades previstas na segunda etapa do trabalho, representando-o em forma de cronograma na Figura \ref{fig:cronograma_2}.


\begin{figure}[h!]
    \caption{Cronograma das Atividades na Segunda Etapa da Monografia}
    \centering
    \includegraphics[width=.7\linewidth]{figuras/cronograma2.png}
    \caption*{Fonte: Autor}
    \label{fig:cronograma_2}
\end{figure}


\begin{itemize}
    \item \textbf{Evolução do Trabalho orientado às Observações da Banca:} realização dos ajustes, melhorias estruturais e refinamentos conceituais decorrentes dos apontamentos feitos pela banca avaliadora na etapa anterior, com foco em aprimorar clareza, rigor metodológico e fundamentação do estudo;
    \item \textbf{Preparação e Organização dos Artefatos:} construção e organização das especificações BDD, inputs textuais, configurações do pipeline e demais materiais necessários para conduzir a avaliação empírica da solução. Essa etapa assegura que todos os artefatos estejam consistentes com o protocolo planejado;
    \item \textbf{Execução do Estudo Observacional:} realização da coleta de dados por meio da execução completa do pipeline da solução. Nesta etapa são registradas evidências, falhas, padrões e comportamentos relevantes;
    \item \textbf{Análise e Consolidação dos Resultados:} interpretação dos dados coletados, identificação de padrões, comportamentos e problemas observados, seguida da resposta às questões de pesquisa e da documentação dos principais resultados obtidos;
    \item \textbf{Escrita da Monografia:} elaboração das seções referentes ao estudo observacional, incluindo apresentação dos resultados, discussão, limitações, ameaças à validade e considerações finais.
    \item \textbf{Revisão da Monografia:} realização das correções e ajustes solicitados pelo orientador;
    \item \textbf{Apresentação da Monografia:} preparação e apresentação da monografia finalizada para a banca avaliadora.
\end{itemize}